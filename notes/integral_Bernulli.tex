\section{Интеграл Бернулли}
Рассмотрим стационарную систему Навье-Стокса:
\begin{equation*}
	\left\{
		\begin{array}{l}
		\sum\limits_{i=1}^{3}V_i\frac{\partial V}{\partial x_i}
		+ \grad p - \mu \Delta V = f
		\\
		\diver V = 0
		\end{array}
	\right.
\end{equation*}

Напомним, что $\mu$ в ней --- вязкость жидкости;
жидкость, вязкость которой равна нулю, называется идеальной.
Стационарная система Навье-Стокса для идеальной жидкости называется стационарной системой Эйлера:
\begin{equation} \label{Eiler_stac_int_Bernulli}
	\left\{
		\begin{array}{l}
		\sum\limits_{i=1}^{3}V_i\frac{\partial V}{\partial x_i}
		+ \grad p = f
		\\
		\diver V = 0
		\end{array}
	\right.
\end{equation}

Эта система в некотором смысле сложнее системы Навье-Стокса,
так как имеет качественно иную (не параболическую, а гиперболичскую) природу.

Предположим, что закон движения $x=u(t,X)$ нам известен.
Умножим первое уравнение системы \eqref{Eiler_stac_int_Bernulli} на $V(x) = V(u(t,X))$:
\begin{equation} \label{Eiler_stac_int_Bernulli_scal}
	\sum\limits_{i=1}^{3}\left<V_i\frac{\partial V}{\partial x_i} , V \right>
	+ \left< \grad p , V \right> = \left< f , V \right>
\end{equation}

Предположим далее, что поле сил $f$ потенциально, т.~е. $f = - \nabla G$.

Несложно доказать (см. \eqref{d-dt-V-2}), что
$\sum\limits_{i=1}^{3} \left< V_i \frac{\partial V}{\partial x_i}, V \right> = \frac{1}{2} \frac{d}{dt}V^2$,
а также (см. \eqref{grad-phi-V}), что
$\left< \nabla p , V\right> = \frac{d}{dt}p (u(t,X))$
и $\left< f , V\right> = - \frac{d}{dt}G (u(t,X))$.

Тогда уравнение \eqref{Eiler_stac_int_Bernulli_scal} принимает вид
\begin{equation*}
	\frac{1}{2} \frac{d}{dt}V^2 + \frac{d}{dt}p (u(t,X)) + \frac{d}{dt}G (u(t,X)) = 0
\end{equation*}

Или, что то же самое,
\begin{equation*} \label{Eiler_stac_int_Bernulli_fulldiff}
	\frac{d}{dt}\left(\frac{1}{2} V^2(u(t,X)) + p(u(t,X)) + G (u(t,X))\right) = 0
\end{equation*}
Из уравнения \eqref{Eiler_stac_int_Bernulli_fulldiff} следует, что выражение,
стоящее под знаком дифференциала, не зависит от $t$ (но может зависеть от $x$), т.~е.
\begin{equation*} \label{int_Bernulli}
	\frac{1}{2} V^2(x) + p(x) + G (x) = C(x)
\end{equation*}
Это выражение и называется интегралом Бернулли.

Например, если на частицы жидкости действует только сила тяжести,
т.~е. $G(x) = -gx_3$, то интеграл Бернулли принимает вид
\begin{equation*} \label{int_Bernulli}
	\frac{1}{2} V^2(x) + p(x) + G (x_3) = C(x)
\end{equation*}
Из этой записи, в частности, можно сделать вывод,
что при постоянной высоте $x_3$ чем выше скорость жидкости в точке,
тем ниже давление в этой точке.
