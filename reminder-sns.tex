\paragraph{Система Навье-Стокса в наиболее общем случае}
$$
	\left\{
		\begin{array}{l}
		\rho\left(
			\frac{\partial V}{\partial t}+\sum\limits_{i=1}^{3}V_i\frac{\partial V}{\partial x_i}
		\right) + \grad p - \mu \Delta V = \rho f
		\\
		\diver V = 0
		\end{array}
	\right.
$$
Здесь
$x = (x_1; x_2; x_3)$ --- координата частицы,
$V = V(t,x)$ --- скорость частицы с координатой $x$ в момент времени $t$,
$p = p(t,x)$ --- функция давления,
$\rho = \rho(t,x)$ --- функция давления,
$\mu = const$ --- вязкость,
$f = f(t,x)$ ---вектор-функция сил.
Все дифференциальные операторы (дивергенция, градиент, лапласиан)
действуют по переменной $x$.

\paragraph{Система Навье-Стокса c постоянной (единичной) плотностью}
$$
	\left\{
		\begin{array}{l}
		\frac{\partial V}{\partial t}+\sum\limits_{i=1}^{3}V_i\frac{\partial V}{\partial x_i}
		+ \grad p - \mu \Delta V = f
		\\
		\diver V = 0
		\end{array}
	\right.
$$


\paragraph{Стационарная система Навье-Стокса c постоянной (единичной) плотностью}
$$
	\left\{
		\begin{array}{l}
		\sum\limits_{i=1}^{3}V_i\frac{\partial V}{\partial x_i}
		+ \grad p - \mu \Delta V = f
		\\
		\diver V = 0
		\end{array}
	\right.
$$
