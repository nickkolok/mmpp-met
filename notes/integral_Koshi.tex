\section{Интеграл Коши}

Интеграл Коши --- достаточно близкий аналог интеграла Бернулли.
Рассмотрим стационарную систему Эйлера \eqref{Eiler_stac_int_Bernulli},
сделав два предположения о потенциальности:
\begin{equation} \label{int_Koshi_potenc}
	f = -\nabla G,~~~~~~ V = \nabla \varphi,
\end{equation}
т.~е. потребуем, в отличие от вывода интеграла Бернулли,
чтобы потенциальным было не только поле сил, но и поле скоростей.

Заметим, что интеграл Бернулли получается в результате приведения первого уравнения системы
\eqref{Eiler_stac_int_Bernulli} к полному дифференциалу по $t$;
в нашем же случае логично приводить его к виду
$$
	\nabla(\dots) = 0
$$
Несложно доказать (см. \eqref{1-2-nabla-V-2}), что
$$
	\sum\limits_{j=1}^n V_j \frac{\partial V_j}{\partial x_i} = \frac{1}{2}\nabla(V^2)_i
$$

Таким образом, первое уравнение системы \eqref{Eiler_stac_int_Bernulli} с учётом \eqref{int_Koshi_potenc}
принимает вид
$$
	\nabla \left(
		\frac{1}{2}V^2 + p + G = C
	\right) = 0
$$
Это уравнение и называется интегралом Коши.
Заметим, что здесь константа $C$ не зависит ни от времени, ни от координаты.

В частности, возвращаясь к примеру с силой тяжести ($G(x) = -gx_3$),
мы можем с помощью интеграла Коши сделать вывод,
что на любой высоте (не обязательно постоянной) давление в жидкости тем меньше,
чем выше скорость её движения.

Подчеркнём, что усиление результата достигнуто ценою введение дополнительного требования ---
требования потенциальности поля скоростей.
