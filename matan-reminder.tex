\paragraph{Дивергенция}
--- это сумма частных производных:
$$
	\diver u(x_1, \dots, x_n) =
	\sum_{i=1}^n \frac{\partial u}{\partial x_n} =
	\frac{\partial u}{\partial x_1} + \frac{\partial u}{\partial x_2} +
	\dots + \frac{\partial u}{\partial x_n}
$$

\paragraph{Градиент}
--- это вектор частных производных:
$$
	\grad u(x_1, \dots, x_n) =
	\nabla u(x_1, \dots, x_n) =
	\left( \frac{\partial u}{\partial x_i}\right)_{i=1}^{n} =
	\left( \frac{\partial u}{\partial x_1} ; \frac{\partial u}{\partial x_2} ;
	\dots; \frac{\partial u}{\partial x_n}\right)
$$

\paragraph{Ротор}
(вихрь) векторной функции
$V \colon \R^3 \to \R^3$
мнемонически соотносится с формальным определителем:
$$
	\rot V(x_1, x_2, x_3) =
	\left|
		\begin{array}{ccc}
			e_1                           & e_2                           & e_3 \\
			\frac{\partial}{\partial x_1} & \frac{\partial}{\partial x_2} & \frac{\partial}{\partial x_3} \\
			V_1                           & V_2                           & V_3
		\end{array}
	\right|
	=
	\left(
		\frac{\partial V_3}{\partial x_2} - \frac{\partial V_2}{\partial x_3} ;
		\frac{\partial V_1}{\partial x_3} - \frac{\partial V_3}{\partial x_1} ;
		\frac{\partial V_2}{\partial x_1} - \frac{\partial V_1}{\partial x_2}
	\right),
$$
где $e_1$, $e_2$, $e_3$ --- координатные вектора.
